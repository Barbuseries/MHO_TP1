\documentclass[12pt, letterpaper]{article}

\usepackage[T1]{fontenc}
\usepackage[utf8]{inputenc}
\usepackage[frenchb]{babel}
\usepackage{gensymb}
\usepackage{latexsym}
\usepackage{titlesec}
\usepackage{marvosym}
\usepackage{enumitem}
\usepackage[pdftex, hidelinks]{hyperref}

\frenchbsetup{StandardItemLabels=true}

\author{Edorh François, Guison Vianney}
\title{Rapport TP1 de Métaheuristiques en optimisation}


\begin{document}
\maketitle
\tableofcontents
\newpage

\section{Méthodes implémentées}

\subsection{Crossovers}

Toutes les méthodes de crossover sont dans le fichier Crossover.m.
Elles sont stockées dans la variable globale CROSSOVER.

\subsubsection{Binaires}

Les méthodes de crossover sur valeurs binaires suivantes ont été implémentés:
\begin{itemize}
\item single-point crossover\\
  
\item multi-point crossover, avec un paramètre de contrôle N compris
entre 1 et $L - 1$ (L étant la longeur du chromosome)\\
  
\item uniform crossover, avec deux variantes:
  \begin{itemize}
  \item Utiliser un paramètre P représentant une probabilité constante\\
    
  \item Utiliser deux paramètres de contrôle P et T, où la probabilité
    d'une pair (a, b) est donnée par $P(T(a), T(b))$.\\
    
  \end{itemize}
\end{itemize}

\subsubsection{À valeurs réelles}

Les méthodes de crossover sur valeurs réelles suivantes ont été implémentés:

\begin{itemize}
\item whole arithmetic crossover\\
  
\item local arithmetic crossover\\
  
\item blend crossover (ou $BLX-\alpha$), avec un paramètre de contrôle
  $\alpha$ (valeur par défaut de 0.5)\\
  
\item simulated binary crossover, avec un paramètre de contrôle $N >= 0$.\\
\end{itemize}


\subsubsection{Au choix}

oneBitAdaptation(F0, F1), F0 and F1 as crossover functions
TODO: Complete

\subsection{Mutations}

Toutes les méthodes de mutation sont dans le fichier Mutation.m.
Elles sont stockées dans la variable globale MUTATION.

\subsubsection{Binaires}

La méthode de mutation binaire implémentée est la mutation bit-flip.

\subsubsection{À valeurs réelles}

Les méthodes de mutation sur valeurs réelles suivantes ont été implémentés:

\begin{itemize}
\item uniform mutation\\
  
\item boundary mutation\\
  
\item normal mutation, TODO: Check control parameters\\
  
\item normalN mutation, TODO: Check control parameters\\
  
\item polynomial mutation, avec un paramètre de contrôle $N >= 0$\\
  
\item non-uniform mutation, avec un paramètre de contrôle B.\\
\end{itemize}

\subsection{Sélections}

Toutes les méthodes de sélection sont dans le fichier Sélection.m.
Elles sont stockées dans la variable globale SELECTION.

Les méthodes de sélection suivantes ont été implémentés:

\begin{itemize}
  
\item wheel selection\\
	
\item stochastic universal sampling\\
	
\item tournament selection, avec un paramètre de contrôle K, compris
  entre 1 et N, le nombre d'individus\\
	
\item unbiased tournament selection, avec un paramètre de contrôle K,
  compris entre 1 et N, le nombre d'individus\\
	
\item truncation selection, avec un paramètre C compris entre 1 et N,
  le nombre d'individus, où $N/C$ correspond au nombre d'individus
  utilisés dans la sélection.\\
\end{itemize}

\subsection{Stratégie "Steady State" }

\subsection{Critères d'arrêt}

Toutes les méthodes d'arrêt sont dans le fichier StopCriteria.m.
Elles sont stockées dans la variable globale STOP\_CRITERIA.

Les critères d'arrêt suivant ont été implémentés:
\begin{itemize}
\item time\\
  
\item threshold, avec deux paramètres de contrôle:
  \begin{itemize}
  \item R, la relation entre la fitness et la limite T ($>=$, $<=$,
    ...)\\
    
  \item T, la limite fixée\\
  \end{itemize}
	
\item variance, avec un paramètre de contrôle V\\
	
\item min-max ratio, avec un paramètre de contrôle R\\
	
\item mean-change rate, avec un paramètre de contrôle CR\\
\end{itemize}

N.B: Même si un critère d'arrêt différent du temps est défini,
l'algorithme ne dépassera pas le nombre d'itérations maximum donnés.

\end{document}
