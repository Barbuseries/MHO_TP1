\documentclass[12pt, letterpaper]{article}

\usepackage[T1]{fontenc}
\usepackage[utf8]{inputenc}
\usepackage[frenchb]{babel}
\usepackage{gensymb}
\usepackage{latexsym}
\usepackage{titlesec}
\usepackage{marvosym}
\usepackage{enumitem}
\usepackage[pdftex, hidelinks]{hyperref}

\author{Edorh François, Guison Vianney}
\title{Rapport TP1 de Métaheuristiques en optimisation}

\maketitle
\tableofcontents
\newpage

\begin{document}

\section{Méthodes implémentées}

\subsection{Crossovers}

\subsubsection{Binaires}

Les méthodes de crossovers sur valeurs binaires suivant ont été implémentés:
\begin{itemize}
\item single-point crossover\\
  
\item multi-point crossover, avec un paramètre de contrôle N compris
entre 1 et L - 1 (L étant la longeur du chromosome)\\
  
\item uniform crossover, avec deux variantes:
  \begin{itemize}
  \item Utiliser un paramètre P représentant une probabilité constante\\
    
  \item Utiliser deux paramètres de contrôle P et T, où la probabilité
    d'une pair (a, b) est donnée par $P(T(a), T(b))$.\\
    
  \end{itemize}
\end{itemize}

\subsubsection{À valeurs réelles}
Les méthodes de crossovers sur valeurs réelles suivant ont été implémentés:
\begin{itemize}
\item whole arithmetic crossover\\
  
\item local arithmetic crossover\\
  
\item blend crossover (ou $BLX-\alpha$), avec un paramètre de contrôle
  $\alpha$ (valeur par défaut de 0.5)\\
  
\item simulated binary crossover, avec un paramètre de contrôle $N >= 0$.\\
\end{itemize}


\subsubsection{Au choix}
oneBitAdaptation(F0, F1), F0 and F1 as crossover functions
TODO: Complete

\subsection{Mutations}
\subsubsection{Binaires}

\subsubsection{À valeurs réelles}

\subsection{Sélections}

\subsection{Stratégie "Steady State" }

\subsection{Critères d'arrêt}


\end{document}
